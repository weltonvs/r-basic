% Options for packages loaded elsewhere
\PassOptionsToPackage{unicode}{hyperref}
\PassOptionsToPackage{hyphens}{url}
%
\documentclass[
  ignorenonframetext,
]{beamer}
\usepackage{pgfpages}
\setbeamertemplate{caption}[numbered]
\setbeamertemplate{caption label separator}{: }
\setbeamercolor{caption name}{fg=normal text.fg}
\beamertemplatenavigationsymbolsempty
% Prevent slide breaks in the middle of a paragraph
\widowpenalties 1 10000
\raggedbottom
\setbeamertemplate{part page}{
  \centering
  \begin{beamercolorbox}[sep=16pt,center]{part title}
    \usebeamerfont{part title}\insertpart\par
  \end{beamercolorbox}
}
\setbeamertemplate{section page}{
  \centering
  \begin{beamercolorbox}[sep=12pt,center]{part title}
    \usebeamerfont{section title}\insertsection\par
  \end{beamercolorbox}
}
\setbeamertemplate{subsection page}{
  \centering
  \begin{beamercolorbox}[sep=8pt,center]{part title}
    \usebeamerfont{subsection title}\insertsubsection\par
  \end{beamercolorbox}
}
\AtBeginPart{
  \frame{\partpage}
}
\AtBeginSection{
  \ifbibliography
  \else
    \frame{\sectionpage}
  \fi
}
\AtBeginSubsection{
  \frame{\subsectionpage}
}
\usepackage{lmodern}
\usepackage{amssymb,amsmath}
\usepackage{ifxetex,ifluatex}
\ifnum 0\ifxetex 1\fi\ifluatex 1\fi=0 % if pdftex
  \usepackage[T1]{fontenc}
  \usepackage[utf8]{inputenc}
  \usepackage{textcomp} % provide euro and other symbols
\else % if luatex or xetex
  \usepackage{unicode-math}
  \defaultfontfeatures{Scale=MatchLowercase}
  \defaultfontfeatures[\rmfamily]{Ligatures=TeX,Scale=1}
\fi
% Use upquote if available, for straight quotes in verbatim environments
\IfFileExists{upquote.sty}{\usepackage{upquote}}{}
\IfFileExists{microtype.sty}{% use microtype if available
  \usepackage[]{microtype}
  \UseMicrotypeSet[protrusion]{basicmath} % disable protrusion for tt fonts
}{}
\makeatletter
\@ifundefined{KOMAClassName}{% if non-KOMA class
  \IfFileExists{parskip.sty}{%
    \usepackage{parskip}
  }{% else
    \setlength{\parindent}{0pt}
    \setlength{\parskip}{6pt plus 2pt minus 1pt}}
}{% if KOMA class
  \KOMAoptions{parskip=half}}
\makeatother
\usepackage{xcolor}
\IfFileExists{xurl.sty}{\usepackage{xurl}}{} % add URL line breaks if available
\IfFileExists{bookmark.sty}{\usepackage{bookmark}}{\usepackage{hyperref}}
\hypersetup{
  pdftitle={Tema 2 - Documentación con R Markdown},
  pdfauthor={Juan Gabriel Gomila \& María Santos},
  hidelinks,
  pdfcreator={LaTeX via pandoc}}
\urlstyle{same} % disable monospaced font for URLs
\newif\ifbibliography
\usepackage{color}
\usepackage{fancyvrb}
\newcommand{\VerbBar}{|}
\newcommand{\VERB}{\Verb[commandchars=\\\{\}]}
\DefineVerbatimEnvironment{Highlighting}{Verbatim}{commandchars=\\\{\}}
% Add ',fontsize=\small' for more characters per line
\usepackage{framed}
\definecolor{shadecolor}{RGB}{248,248,248}
\newenvironment{Shaded}{\begin{snugshade}}{\end{snugshade}}
\newcommand{\AlertTok}[1]{\textcolor[rgb]{0.94,0.16,0.16}{#1}}
\newcommand{\AnnotationTok}[1]{\textcolor[rgb]{0.56,0.35,0.01}{\textbf{\textit{#1}}}}
\newcommand{\AttributeTok}[1]{\textcolor[rgb]{0.77,0.63,0.00}{#1}}
\newcommand{\BaseNTok}[1]{\textcolor[rgb]{0.00,0.00,0.81}{#1}}
\newcommand{\BuiltInTok}[1]{#1}
\newcommand{\CharTok}[1]{\textcolor[rgb]{0.31,0.60,0.02}{#1}}
\newcommand{\CommentTok}[1]{\textcolor[rgb]{0.56,0.35,0.01}{\textit{#1}}}
\newcommand{\CommentVarTok}[1]{\textcolor[rgb]{0.56,0.35,0.01}{\textbf{\textit{#1}}}}
\newcommand{\ConstantTok}[1]{\textcolor[rgb]{0.00,0.00,0.00}{#1}}
\newcommand{\ControlFlowTok}[1]{\textcolor[rgb]{0.13,0.29,0.53}{\textbf{#1}}}
\newcommand{\DataTypeTok}[1]{\textcolor[rgb]{0.13,0.29,0.53}{#1}}
\newcommand{\DecValTok}[1]{\textcolor[rgb]{0.00,0.00,0.81}{#1}}
\newcommand{\DocumentationTok}[1]{\textcolor[rgb]{0.56,0.35,0.01}{\textbf{\textit{#1}}}}
\newcommand{\ErrorTok}[1]{\textcolor[rgb]{0.64,0.00,0.00}{\textbf{#1}}}
\newcommand{\ExtensionTok}[1]{#1}
\newcommand{\FloatTok}[1]{\textcolor[rgb]{0.00,0.00,0.81}{#1}}
\newcommand{\FunctionTok}[1]{\textcolor[rgb]{0.00,0.00,0.00}{#1}}
\newcommand{\ImportTok}[1]{#1}
\newcommand{\InformationTok}[1]{\textcolor[rgb]{0.56,0.35,0.01}{\textbf{\textit{#1}}}}
\newcommand{\KeywordTok}[1]{\textcolor[rgb]{0.13,0.29,0.53}{\textbf{#1}}}
\newcommand{\NormalTok}[1]{#1}
\newcommand{\OperatorTok}[1]{\textcolor[rgb]{0.81,0.36,0.00}{\textbf{#1}}}
\newcommand{\OtherTok}[1]{\textcolor[rgb]{0.56,0.35,0.01}{#1}}
\newcommand{\PreprocessorTok}[1]{\textcolor[rgb]{0.56,0.35,0.01}{\textit{#1}}}
\newcommand{\RegionMarkerTok}[1]{#1}
\newcommand{\SpecialCharTok}[1]{\textcolor[rgb]{0.00,0.00,0.00}{#1}}
\newcommand{\SpecialStringTok}[1]{\textcolor[rgb]{0.31,0.60,0.02}{#1}}
\newcommand{\StringTok}[1]{\textcolor[rgb]{0.31,0.60,0.02}{#1}}
\newcommand{\VariableTok}[1]{\textcolor[rgb]{0.00,0.00,0.00}{#1}}
\newcommand{\VerbatimStringTok}[1]{\textcolor[rgb]{0.31,0.60,0.02}{#1}}
\newcommand{\WarningTok}[1]{\textcolor[rgb]{0.56,0.35,0.01}{\textbf{\textit{#1}}}}
\usepackage{longtable,booktabs}
\usepackage{caption}
% Make caption package work with longtable
\makeatletter
\def\fnum@table{\tablename~\thetable}
\makeatother
\setlength{\emergencystretch}{3em} % prevent overfull lines
\providecommand{\tightlist}{%
  \setlength{\itemsep}{0pt}\setlength{\parskip}{0pt}}
\setcounter{secnumdepth}{-\maxdimen} % remove section numbering

\title{Tema 2 - Documentación con R Markdown}
\author{Juan Gabriel Gomila \& María Santos}
\date{}

\begin{document}
\frame{\titlepage}

\hypertarget{introducciuxf3n}{%
\section{Introducción}\label{introducciuxf3n}}

\begin{frame}{Markdown}
\protect\hypertarget{markdown}{}

R Markdown. Es un tipo de fichero en el cual podemos intercalar sin
problema alguno texto, código y fórmulas matemáticas.

Para la mayor parte de las necesidades de este curso, en lo referente a
la creación y composición de este tipo de ficheros, el documento
\emph{\href{https://en.support.wordpress.com/markdown-quick-reference/}{Markdown
Quick Reference}} y la
\href{https://www.rstudio.com/wp-content/uploads/2015/02/rmarkdown-cheatsheet.pdf}{chuleta}
de R Markdown deberían ser suficientes.

Sin embargo, a lo largo de este curso iremos ampliando estos contenidos
en algunos temas cuando lo creamos necesario.

Nosotros, en este tema, veremos cómo controlar el comportamiento de los
bloques de código (chunks) al compilar el fichero R Markdown y cómo
escribir fórmulas matemáticas bien formateadas.

\end{frame}

\hypertarget{fuxf3rmulas-matemuxe1ticas}{%
\section{Fórmulas matemáticas}\label{fuxf3rmulas-matemuxe1ticas}}

\begin{frame}[fragile]{Cómo escribir}
\protect\hypertarget{cuxf3mo-escribir}{}

Para escribir fórmulas matemáticas bien formateadas utilizaremos la
sintaxis \(\LaTeX\)

\begin{itemize}
\tightlist
\item
  Para tener ecuaciones o fórmulas en el mismo párrafo, escribimos
  nuestro código entre dos símbolos de dólar:
  \texttt{\$}código\texttt{\$}
\item
  Si queremos tener ecuaciones o fórmulas centradas en un párrafo
  aparte, escribimos nuestro código entre dos dobles símbolos de dólar:
  \texttt{\$\$}código\texttt{\$\$}
\end{itemize}

¡Cuidado! Al escribir una fórmula de la forma indicada anteriormente o
simplemente texto en R Markdown, los espacios en blanco son
completamente ignorados. RStudio solamente añade los espacios en blanco
a partir del significado lógico de sus elementos.

\end{frame}

\begin{frame}[fragile]{Espacios en blanco}
\protect\hypertarget{espacios-en-blanco}{}

\textbf{Ejemplo}

Para que veáis que RStudio ignora el exceso de espacios en blanco, aquí
os damos un ejemplo en el que hemos introducido espacios innecesarios:

Código: \texttt{En\ en\ instituto}\(\ \ \ \ \ \)
\texttt{nos\ enseñaron\ que\ las\ raíces\ de\ las\ ecuaciones\ de\ tercer\ grado,\ de\ la\ forma\ \$Ax\^{}3+Bx\^{}2+Cx+D=0\$,\ se\ encuentran\ mediante\ \textbackslash{}textit\{la\ Regla\ de\ Ruffini\}.\ Por\ su\ parte,}\(\ \ \ \ \ \ \ \ \ \)
\texttt{las\ raíces\ de\ las\ ecuaciones\ de\ segundo\ grado\ de\ la\ forma\ \$\textbackslash{}alpha\ x\^{}2+\textbackslash{}beta\ x+\textbackslash{}gamma=0\$\ se\ encuentran\ siguiendo\ la\ fórmula\ \$\$x\ =\ \textbackslash{}frac\{-\textbackslash{}beta\textbackslash{}pm\textbackslash{}sqrt\{\textbackslash{}beta\^{}2}\(\ \ \ \ \ \ \ \ \ \ \ \ \)\texttt{-4\textbackslash{}alpha\textbackslash{}gamma\}\}\{2\textbackslash{}alpha\}\$\$.}

Resultado: En en instituto nos enseñaron que las raíces de las
ecuaciones de tercer grado, de la forma \(Ax^3+Bx^2+Cx+D=0\), se
encuentran mediante \emph{la Regla de Ruffini}. Por su parte, las raíces
de las ecuaciones de segundo grado de la forma
\(\alpha x^2+\beta x+\gamma=0\) se encuentran siguiendo la fórmula
\[x = \frac{-\beta\pm\sqrt{\beta^2    -4\alpha\gamma}}{2\alpha}\].

\end{frame}

\begin{frame}{Símbolos}
\protect\hypertarget{suxedmbolos}{}

Hay muchísimos símbolos matemáticos que puedes escribirse con la
sintaxis \(\LaTeX\). En el ejemplo anterior ya os hemos mostrado unos
pocos. En este tema, nosotros solo veremos los más utilizados.

Para quien quiera ir más allá, aquí os dejamos un
\href{http://www.ptep-online.com/ctan/symbols.pdf}{documento muy útil}
con gran cantidad de símbolos de \(\LaTeX\).

\end{frame}

\begin{frame}[fragile]{Símbolos matemáticos - Básico}
\protect\hypertarget{suxedmbolos-matemuxe1ticos---buxe1sico}{}

\begin{longtable}[]{@{}lll@{}}
\toprule
Significado & Código & Resultado\tabularnewline
\midrule
\endhead
Suma & \texttt{+} & \(+\)\tabularnewline
Resta & \texttt{-} & \(-\)\tabularnewline
Producto & \texttt{\textbackslash{}cdot} & \(\cdot\)\tabularnewline
Producto & \texttt{\textbackslash{}times} & \(\times\)\tabularnewline
División & \texttt{\textbackslash{}div} & \(\div\)\tabularnewline
Potencia & \texttt{a\^{}\{x\}} & \(a^{x}\)\tabularnewline
Subíndice & \texttt{a\_\{i\}} & \(a_{i}\)\tabularnewline
\bottomrule
\end{longtable}

\end{frame}

\begin{frame}[fragile]{Símbolos matemáticos - Básico}
\protect\hypertarget{suxedmbolos-matemuxe1ticos---buxe1sico-1}{}

\begin{longtable}[]{@{}lll@{}}
\toprule
Significado & Código & Resultado\tabularnewline
\midrule
\endhead
Fracción & \texttt{\textbackslash{}frac\{a\}\{b\}} &
\(\frac{a}{b}\)\tabularnewline
Más menos & \texttt{\textbackslash{}pm} & \(\pm\)\tabularnewline
Raíz n-ésima & \texttt{\textbackslash{}sqrt{[}n{]}\{x\}} &
\(\sqrt[n]{x}\)\tabularnewline
Unión & \texttt{\textbackslash{}cup} & \(\cup\)\tabularnewline
Intersección & \texttt{\textbackslash{}cap} & \(\cap\)\tabularnewline
OR lógico & \texttt{\textbackslash{}vee} & \(\vee\)\tabularnewline
AND lógico & \texttt{\textbackslash{}wedge} & \(\wedge\)\tabularnewline
\bottomrule
\end{longtable}

\end{frame}

\begin{frame}[fragile]{Símbolos matemáticos - Relaciones}
\protect\hypertarget{suxedmbolos-matemuxe1ticos---relaciones}{}

\begin{longtable}[]{@{}lll@{}}
\toprule
Significado & Código & Resultado\tabularnewline
\midrule
\endhead
Igual & \texttt{=} & \(=\)\tabularnewline
Aproximado & \texttt{\textbackslash{}approx} &
\(\approx\)\tabularnewline
No igual & \texttt{\textbackslash{}ne} & \(\ne\)\tabularnewline
Mayor que & \texttt{\textgreater{}} & \(>\)\tabularnewline
Menor que & \texttt{\textless{}} & \(<\)\tabularnewline
Mayor o igual que & \texttt{\textbackslash{}ge} & \(\ge\)\tabularnewline
Menor o igual que & \texttt{\textbackslash{}le} & \(\le\)\tabularnewline
\bottomrule
\end{longtable}

\end{frame}

\begin{frame}[fragile]{Símbolos matemáticos - Operadores}
\protect\hypertarget{suxedmbolos-matemuxe1ticos---operadores}{}

\begin{longtable}[]{@{}lll@{}}
\toprule
Significado & Código & Resultado\tabularnewline
\midrule
\endhead
Sumatorio & \texttt{\textbackslash{}sum\_\{i=0\}\^{}\{n\}} &
\(\sum_{i=0}^{n}\)\tabularnewline
Productorio & \texttt{\textbackslash{}prod\_\{i=0\}\^{}\{n\}} &
\(\prod_{i=0}^{n}\)\tabularnewline
Integral & \texttt{\textbackslash{}int\_\{a\}\^{}\{b\}} &
\(\int_{a}^{b}\)\tabularnewline
Unión (grande) & \texttt{\textbackslash{}bigcup} &
\(\bigcup\)\tabularnewline
Intersección (grande) & \texttt{\textbackslash{}bigcap} &
\(\bigcap\)\tabularnewline
OR lógico (grande) & \texttt{\textbackslash{}bigvee} &
\(\bigvee\)\tabularnewline
AND lógico (grande) & \texttt{\textbackslash{}bigwedge} &
\(\bigwedge\)\tabularnewline
\bottomrule
\end{longtable}

\end{frame}

\begin{frame}[fragile]{Símbolos matemáticos - Delimitadores}
\protect\hypertarget{suxedmbolos-matemuxe1ticos---delimitadores}{}

\begin{longtable}[]{@{}lll@{}}
\toprule
Significado & Código & Resultado\tabularnewline
\midrule
\endhead
Paréntesis & \texttt{()} & \((\ )\)\tabularnewline
Corchetes & \texttt{{[}{]}} & \([\ ]\)\tabularnewline
Llaves & \texttt{\textbackslash{}\{\ \textbackslash{}\}} &
\(\{\ \}\)\tabularnewline
Diamante & \texttt{\textbackslash{}langle\ \textbackslash{}rangle} &
\(\langle\ \rangle\)\tabularnewline
Parte entera por defecto &
\texttt{\textbackslash{}lfloor\ \textbackslash{}rfloor} &
\(\lfloor\  \rfloor\)\tabularnewline
Parte entera por exceso &
\texttt{\textbackslash{}lceil\ \textbackslash{}rceil} &
\(\lceil\ \rceil\)\tabularnewline
Espacio en blanco & \texttt{hola\textbackslash{}\ caracola} &
\(hola\ caracola\)\tabularnewline
\bottomrule
\end{longtable}

\end{frame}

\begin{frame}[fragile]{Símbolos matemáticos - Letras griegas}
\protect\hypertarget{suxedmbolos-matemuxe1ticos---letras-griegas}{}

\begin{longtable}[]{@{}lll@{}}
\toprule
Significado & Código & Resultado\tabularnewline
\midrule
\endhead
Alpha & \texttt{\textbackslash{}alpha} & \(\alpha\)\tabularnewline
Beta & \texttt{\textbackslash{}beta} & \(\beta\)\tabularnewline
Gamma & \texttt{\textbackslash{}gamma\ \textbackslash{}Gamma} &
\(\gamma\  \Gamma\)\tabularnewline
Delta & \texttt{\textbackslash{}delta\ \textbackslash{}Delta} &
\(\delta\  \Delta\)\tabularnewline
Epsilon & \texttt{\textbackslash{}epsilon} & \(\epsilon\)\tabularnewline
Epsilon & \texttt{\textbackslash{}varepsilon} &
\(\varepsilon\)\tabularnewline
Zeta & \texttt{\textbackslash{}zeta} & \(\zeta\)\tabularnewline
\bottomrule
\end{longtable}

\end{frame}

\begin{frame}[fragile]{Símbolos matemáticos - Letras griegas}
\protect\hypertarget{suxedmbolos-matemuxe1ticos---letras-griegas-1}{}

\begin{longtable}[]{@{}lll@{}}
\toprule
Significado & Código & Resultado\tabularnewline
\midrule
\endhead
Eta & \texttt{\textbackslash{}eta} & \(\eta\)\tabularnewline
Theta & \texttt{\textbackslash{}theta\ \textbackslash{}Theta} &
\(\theta\ \Theta\)\tabularnewline
Kappa & \texttt{\textbackslash{}kappa} & \(\kappa\)\tabularnewline
Lambda & \texttt{\textbackslash{}lambda\ \textbackslash{}Lambda} &
\(\lambda\  \Lambda\)\tabularnewline
Mu & \texttt{\textbackslash{}mu} & \(\mu\)\tabularnewline
Nu & \texttt{\textbackslash{}nu} & \(\nu\)\tabularnewline
Xi & \texttt{\textbackslash{}xi\ \textbackslash{}Xi} &
\(\xi\ \Xi\)\tabularnewline
\bottomrule
\end{longtable}

\end{frame}

\begin{frame}[fragile]{Símbolos matemáticos - Letras griegas}
\protect\hypertarget{suxedmbolos-matemuxe1ticos---letras-griegas-2}{}

\begin{longtable}[]{@{}lll@{}}
\toprule
Significado & Código & Resultado\tabularnewline
\midrule
\endhead
Pi & \texttt{\textbackslash{}pi\ \textbackslash{}Pi} &
\(\pi\ \Pi\)\tabularnewline
Rho & \texttt{\textbackslash{}rho} & \(\rho\)\tabularnewline
Sigma & \texttt{\textbackslash{}sigma\ \textbackslash{}Sigma} &
\(\sigma\ \Sigma\)\tabularnewline
Tau & \texttt{\textbackslash{}tau} & \(\tau\)\tabularnewline
Upsilon & \texttt{\textbackslash{}upsilon\ \textbackslash{}Upsilon} &
\(\upsilon\ \Upsilon\)\tabularnewline
Phi & \texttt{\textbackslash{}phi\ \textbackslash{}Phi} &
\(\phi\ \Phi\)\tabularnewline
Phi & \texttt{\textbackslash{}varphi} & \(\varphi\)\tabularnewline
\bottomrule
\end{longtable}

\end{frame}

\begin{frame}[fragile]{Símbolos matemáticos - Letras griegas}
\protect\hypertarget{suxedmbolos-matemuxe1ticos---letras-griegas-3}{}

\begin{longtable}[]{@{}lll@{}}
\toprule
Significado & Código & Resultado\tabularnewline
\midrule
\endhead
Chi & \texttt{\textbackslash{}chi} & \(\chi\)\tabularnewline
Psi & \texttt{\textbackslash{}psi\ \textbackslash{}Psi} &
\(\psi\ \Psi\)\tabularnewline
Omega & \texttt{\textbackslash{}omega\ \textbackslash{}Omega} &
\(\omega\ \Omega\)\tabularnewline
\bottomrule
\end{longtable}

\end{frame}

\begin{frame}[fragile]{Símbolos matemáticos - Acentos matemáticos}
\protect\hypertarget{suxedmbolos-matemuxe1ticos---acentos-matemuxe1ticos}{}

\begin{longtable}[]{@{}lll@{}}
\toprule
Significado & Código & Resultado\tabularnewline
\midrule
\endhead
Gorrito & \texttt{\textbackslash{}hat\{x\}} & \(\hat{x}\)\tabularnewline
Barra & \texttt{\textbackslash{}bar\{x\}} & \(\bar{x}\)\tabularnewline
Punto 1 & \texttt{\textbackslash{}dot\{x\}} & \(\dot{x}\)\tabularnewline
Punto 2 & \texttt{\textbackslash{}ddot\{x\}} &
\(\ddot{x}\)\tabularnewline
Punto 3 & \texttt{\textbackslash{}dddot\{x\}} &
\(\dddot{x}\)\tabularnewline
Tilde & \texttt{\textbackslash{}tilde\{x\}} &
\(\tilde{x}\)\tabularnewline
Vector & \texttt{\textbackslash{}vec\{x\}} & \(\vec{x}\)\tabularnewline
\bottomrule
\end{longtable}

\end{frame}

\begin{frame}[fragile]{Símbolos matemáticos - Acentos expansibles}
\protect\hypertarget{suxedmbolos-matemuxe1ticos---acentos-expansibles}{}

\begin{longtable}[]{@{}lll@{}}
\toprule
Significado & Código & Resultado\tabularnewline
\midrule
\endhead
Gorrito & \texttt{\textbackslash{}widehat\{xyz\}} &
\(\widehat{xyz}\)\tabularnewline
Barra & \texttt{\textbackslash{}overline\{xyz\}} &
\(\overline{xyz}\)\tabularnewline
Subrallado & \texttt{\textbackslash{}underline\{xyz\}} &
\(\underline{xyz}\)\tabularnewline
Llave superior & \texttt{\textbackslash{}overbrace\{xyz\}} &
\(\overbrace{xyz}\)\tabularnewline
Llave inferior & \texttt{\textbackslash{}underbrace\{xyz\}} &
\(\underbrace{xyz}\)\tabularnewline
Tilde & \texttt{\textbackslash{}widetilde\{xyz\}} &
\(\widetilde{xyz}\)\tabularnewline
Vector & \texttt{\textbackslash{}overrightarrow\{xyz\}} &
\(\overrightarrow{xyz}\)\tabularnewline
\bottomrule
\end{longtable}

\end{frame}

\begin{frame}[fragile]{Símbolos matemáticos - Flechas}
\protect\hypertarget{suxedmbolos-matemuxe1ticos---flechas}{}

\begin{longtable}[]{@{}lll@{}}
\toprule
\begin{minipage}[b]{0.30\columnwidth}\raggedright
Significado\strut
\end{minipage} & \begin{minipage}[b]{0.30\columnwidth}\raggedright
Código\strut
\end{minipage} & \begin{minipage}[b]{0.30\columnwidth}\raggedright
Resultado\strut
\end{minipage}\tabularnewline
\midrule
\endhead
\begin{minipage}[t]{0.30\columnwidth}\raggedright
Simple\strut
\end{minipage} & \begin{minipage}[t]{0.30\columnwidth}\raggedright
\texttt{\textbackslash{}leftarrow\ \textbackslash{}rightarrow}\strut
\end{minipage} & \begin{minipage}[t]{0.30\columnwidth}\raggedright
\(\leftarrow\ \rightarrow\)\strut
\end{minipage}\tabularnewline
\begin{minipage}[t]{0.30\columnwidth}\raggedright
Doble\strut
\end{minipage} & \begin{minipage}[t]{0.30\columnwidth}\raggedright
\texttt{\textbackslash{}Leftarrow\ \textbackslash{}Rightarrow}\strut
\end{minipage} & \begin{minipage}[t]{0.30\columnwidth}\raggedright
\(\Leftarrow\ \Rightarrow\)\strut
\end{minipage}\tabularnewline
\begin{minipage}[t]{0.30\columnwidth}\raggedright
Simple larga\strut
\end{minipage} & \begin{minipage}[t]{0.30\columnwidth}\raggedright
\texttt{\textbackslash{}longleftarrow\ \textbackslash{}longrightarrow}\strut
\end{minipage} & \begin{minipage}[t]{0.30\columnwidth}\raggedright
\(\longleftarrow\  \longrightarrow\)\strut
\end{minipage}\tabularnewline
\begin{minipage}[t]{0.30\columnwidth}\raggedright
Doble larga\strut
\end{minipage} & \begin{minipage}[t]{0.30\columnwidth}\raggedright
\texttt{\textbackslash{}Longleftarrow\ \textbackslash{}Longrightarrow}\strut
\end{minipage} & \begin{minipage}[t]{0.30\columnwidth}\raggedright
\(\Longleftarrow\ \Longrightarrow\)\strut
\end{minipage}\tabularnewline
\begin{minipage}[t]{0.30\columnwidth}\raggedright
Doble sentido simple\strut
\end{minipage} & \begin{minipage}[t]{0.30\columnwidth}\raggedright
\texttt{\textbackslash{}leftrightarrow}\strut
\end{minipage} & \begin{minipage}[t]{0.30\columnwidth}\raggedright
\(\leftrightarrow\)\strut
\end{minipage}\tabularnewline
\begin{minipage}[t]{0.30\columnwidth}\raggedright
Doble sentido doble\strut
\end{minipage} & \begin{minipage}[t]{0.30\columnwidth}\raggedright
\texttt{\textbackslash{}Leftrightarrow}\strut
\end{minipage} & \begin{minipage}[t]{0.30\columnwidth}\raggedright
\(\Leftrightarrow\)\strut
\end{minipage}\tabularnewline
\bottomrule
\end{longtable}

\end{frame}

\begin{frame}[fragile]{Símbolos matemáticos - Flechas}
\protect\hypertarget{suxedmbolos-matemuxe1ticos---flechas-1}{}

\begin{longtable}[]{@{}lll@{}}
\toprule
Significado & Código & Resultado\tabularnewline
\midrule
\endhead
Doble sentido larga simple & \texttt{\textbackslash{}longleftrightarrow}
& \(\longleftrightarrow\)\tabularnewline
Doble sentido larga doble & \texttt{\textbackslash{}Longleftrightarrow}
& \(\Longleftrightarrow\)\tabularnewline
Mapea & \texttt{\textbackslash{}mapsto} & \(\mapsto\)\tabularnewline
Arriba & \texttt{\textbackslash{}uparrow} & \(\uparrow\)\tabularnewline
Abajo & \texttt{\textbackslash{}downarrow} &
\(\downarrow\)\tabularnewline
\bottomrule
\end{longtable}

\end{frame}

\begin{frame}[fragile]{Símbolos matemáticos - Funciones}
\protect\hypertarget{suxedmbolos-matemuxe1ticos---funciones}{}

\begin{longtable}[]{@{}lll@{}}
\toprule
Significado & Código & Resultado\tabularnewline
\midrule
\endhead
Seno & \texttt{\textbackslash{}sin} & \(\sin\)\tabularnewline
Coseno & \texttt{\textbackslash{}cos} & \(\cos\)\tabularnewline
Tangente & \texttt{\textbackslash{}tan} & \(\tan\)\tabularnewline
Arcoseno & \texttt{\textbackslash{}arcsin} & \(\arcsin\)\tabularnewline
Arcocoseno & \texttt{\textbackslash{}arccos} &
\(\arccos\)\tabularnewline
Arcotangente & \texttt{\textbackslash{}arctan} &
\(\arctan\)\tabularnewline
\bottomrule
\end{longtable}

\end{frame}

\begin{frame}[fragile]{Símbolos matemáticos - Funciones}
\protect\hypertarget{suxedmbolos-matemuxe1ticos---funciones-1}{}

\begin{longtable}[]{@{}lll@{}}
\toprule
Significado & Código & Resultado\tabularnewline
\midrule
\endhead
Exponencial & \texttt{\textbackslash{}exp} & \(\exp\)\tabularnewline
Logaritmo & \texttt{\textbackslash{}log} & \(\log\)\tabularnewline
Logaritmo neperiano & \texttt{\textbackslash{}ln} &
\(\ln\)\tabularnewline
Máximo & \texttt{\textbackslash{}max} & \(\max\)\tabularnewline
Mínimo & \texttt{\textbackslash{}min} & \(\min\)\tabularnewline
Límite & \texttt{\textbackslash{}lim} & \(\lim\)\tabularnewline
\bottomrule
\end{longtable}

\end{frame}

\begin{frame}[fragile]{Símbolos matemáticos - Funciones}
\protect\hypertarget{suxedmbolos-matemuxe1ticos---funciones-2}{}

\begin{longtable}[]{@{}lll@{}}
\toprule
Significado & Código & Resultado\tabularnewline
\midrule
\endhead
Supremo & \texttt{\textbackslash{}sup} & \(\sup\)\tabularnewline
Ínfimo & \texttt{\textbackslash{}inf} & \(\inf\)\tabularnewline
Determinante & \texttt{\textbackslash{}det} & \(\det\)\tabularnewline
Argumento & \texttt{\textbackslash{}arg} & \(\arg\)\tabularnewline
\bottomrule
\end{longtable}

\end{frame}

\begin{frame}[fragile]{Símbolos matemáticos - Otros}
\protect\hypertarget{suxedmbolos-matemuxe1ticos---otros}{}

\begin{longtable}[]{@{}lll@{}}
\toprule
Significado & Código & Resultado\tabularnewline
\midrule
\endhead
Puntos suspensivos bajos & \texttt{\textbackslash{}ldots} &
\(\ldots\)\tabularnewline
Puntos suspensivos centrados & \texttt{\textbackslash{}cdots} &
\(\cdots\)\tabularnewline
Puntos suspensivos verticales & \texttt{\textbackslash{}vdots} &
\(\vdots\)\tabularnewline
Puntos suspensivos diagonales & \texttt{\textbackslash{}ddots} &
\(\ddots\)\tabularnewline
Cuantificador existencial & \texttt{\textbackslash{}exists} &
\(\exists\)\tabularnewline
Cuantificador universal & \texttt{\textbackslash{}forall} &
\(\forall\)\tabularnewline
Infinito & \texttt{\textbackslash{}infty} & \(\infty\)\tabularnewline
\bottomrule
\end{longtable}

\end{frame}

\begin{frame}[fragile]{Símbolos matemáticos - Otros}
\protect\hypertarget{suxedmbolos-matemuxe1ticos---otros-1}{}

\begin{longtable}[]{@{}lll@{}}
\toprule
Significado & Código & Resultado\tabularnewline
\midrule
\endhead
Aleph & \texttt{\textbackslash{}aleph} & \(\aleph\)\tabularnewline
Conjunto vacío & \texttt{\textbackslash{}emptyset} &
\(\emptyset\)\tabularnewline
Negación & \texttt{\textbackslash{}neg} & \(\neg\)\tabularnewline
Barra invertida & \texttt{\textbackslash{}backslash} &
\(\backslash\)\tabularnewline
Dollar & \texttt{\textbackslash{}\$} & \(\$\)\tabularnewline
Porcentaje & \texttt{\textbackslash{}\%} & \(\%\)\tabularnewline
Parcial & \texttt{\textbackslash{}partial} & \(\partial\)\tabularnewline
\bottomrule
\end{longtable}

\end{frame}

\begin{frame}[fragile]{Símbolos matemáticos - Tipos de letra}
\protect\hypertarget{suxedmbolos-matemuxe1ticos---tipos-de-letra}{}

\begin{longtable}[]{@{}lll@{}}
\toprule
Significado & Código & Resultado\tabularnewline
\midrule
\endhead
Negrita & \texttt{\textbackslash{}mathbf\{palabra\}} &
\(\mathbf{palabra}\)\tabularnewline
Negrita & \texttt{\textbackslash{}boldsymbol\{palabra\}} &
\(\boldsymbol{palabra}\)\tabularnewline
Negrita de pizarra & \texttt{\textbackslash{}mathbb\{NZQRC\}} &
\(\mathbb{NZQRC}\)\tabularnewline
Caligráfica & \texttt{\textbackslash{}mathcal\{NZQRC\}} &
\(\mathcal{NZQRC}\)\tabularnewline
Gótica & \texttt{\textbackslash{}mathfrak\{NZQRC\}} &
\(\mathfrak{NZQRC}\)\tabularnewline
\bottomrule
\end{longtable}

\end{frame}

\begin{frame}[fragile]{Observaciones}
\protect\hypertarget{observaciones}{}

\begin{itemize}
\item
  A la hora de componer en el interior de un párrafo una fracción,
  existen dos formas: adaptada al tamaño del
  texto,\texttt{\$\textbackslash{}frac\{a\}\{b\}\$}, que resulta en
  \(\frac{a}{b}\); o a tamaño real,
  \texttt{\$\textbackslash{}dfrac\{a\}\{b\}\$}, que da lugar a
  \(\dfrac{a}{b}\).
\item
  Podemos especificar que los delimitadores se adapten a la altura de la
  expresión que envuelven utilizando \texttt{\textbackslash{}left} y
  \texttt{\textbackslash{}right}. Observad el cambio en el siguiente
  ejemplo: \texttt{\$(\textbackslash{}dfrac\{a\}\{b\})\$} y
  \texttt{\$\textbackslash{}left(\textbackslash{}dfrac\{a\}\{b\}\textbackslash{}right)\$}
  producen, respectivamente \((\dfrac{a}{b})\) y
  \(\left(\dfrac{a}{b}\right)\).
\end{itemize}

\end{frame}

\begin{frame}[fragile]{Matrices}
\protect\hypertarget{matrices}{}

\texttt{\$\$\textbackslash{}begin\{matrix\}\ a\_\{11\}\ \&\ a\_\{12\}\ \&\ a\_\{13\}\textbackslash{}\textbackslash{}\ a\_\{21\}\ \&\ a\_\{22\}\ \&\ a\_\{23\}\ \textbackslash{}end\{matrix\}\$\$}

\[\begin{matrix}
a_{11} & a_{12} & a_{13}\\
a_{21} & a_{22} & a_{23}
\end{matrix}\]

\texttt{\$\$\textbackslash{}begin\{pmatrix\}\ a\_\{11\}\ \&\ a\_\{12\}\ \&\ a\_\{13\}\textbackslash{}\textbackslash{}\ a\_\{21\}\ \&\ a\_\{22\}\ \&\ a\_\{23\}\ \textbackslash{}end\{pmatrix\}\$\$}

\[\begin{pmatrix}
a_{11} & a_{12} & a_{13}\\
a_{21} & a_{22} & a_{23}
\end{pmatrix}\]

\end{frame}

\begin{frame}[fragile]{Matrices}
\protect\hypertarget{matrices-1}{}

\texttt{\$\$\textbackslash{}begin\{vmatrix\}\ a\_\{11\}\ \&\ a\_\{12\}\ \&\ a\_\{13\}\textbackslash{}\textbackslash{}\ a\_\{21\}\ \&\ a\_\{22\}\ \&\ a\_\{23\}\ \textbackslash{}end\{vmatrix\}\$\$}

\[\begin{vmatrix}
a_{11} & a_{12} & a_{13}\\
a_{21} & a_{22} & a_{23}
\end{vmatrix}\]

\texttt{\$\$\textbackslash{}begin\{bmatrix\}\ a\_\{11\}\ \&\ a\_\{12\}\ \&\ a\_\{13\}\textbackslash{}\textbackslash{}\ a\_\{21\}\ \&\ a\_\{22\}\ \&\ a\_\{23\}\ \textbackslash{}end\{bmatrix\}\$\$}

\[\begin{bmatrix}
a_{11} & a_{12} & a_{13}\\
a_{21} & a_{22} & a_{23}
\end{bmatrix}\]

\end{frame}

\begin{frame}[fragile]{Matrices}
\protect\hypertarget{matrices-2}{}

\texttt{\$\$\textbackslash{}begin\{Bmatrix\}\ a\_\{11\}\ \&\ a\_\{12\}\ \&\ a\_\{13\}\textbackslash{}\textbackslash{}\ a\_\{21\}\ \&\ a\_\{22\}\ \&\ a\_\{23\}\ \textbackslash{}end\{Bmatrix\}\$\$}

\[\begin{Bmatrix}
a_{11} & a_{12} & a_{13}\\
a_{21} & a_{22} & a_{23}
\end{Bmatrix}\]

\texttt{\$\$\textbackslash{}begin\{Vmatrix\}\ a\_\{11\}\ \&\ a\_\{12\}\ \&\ a\_\{13\}\textbackslash{}\textbackslash{}\ a\_\{21\}\ \&\ a\_\{22\}\ \&\ a\_\{23\}\ \textbackslash{}end\{Vmatrix\}\$\$}

\[\begin{Vmatrix}
a_{11} & a_{12} & a_{13}\\
a_{21} & a_{22} & a_{23}
\end{Vmatrix}\]

\end{frame}

\begin{frame}[fragile]{Sistema de ecuaciones}
\protect\hypertarget{sistema-de-ecuaciones}{}

\texttt{\textbackslash{}begin\{array\}\{l\}\textbackslash{}end\{array\}}
nos produce una tabla alineada a la izquierda. El hecho de introducir el
código \texttt{\textbackslash{}left.\ \textbackslash{}right.} hace que
el delimitador respectivo no aparezca.

\texttt{\$\$\textbackslash{}left.\textbackslash{}begin\{array\}\{l\}\ ax+by=c\textbackslash{}\textbackslash{}\ ex-fy=g\ \textbackslash{}end\{array\}\textbackslash{}right\textbackslash{}\}\$\$}

\[\left.\begin{array}{l}
ax+by=c\\
ex-fy=g
\end{array}\right\}\]

\texttt{\$\$\textbar{}x\textbar{}=\textbackslash{}left\textbackslash{}\{\textbackslash{}begin\{array\}\{rr\}\ -x\ \&\ \textbackslash{}text\{si\ \}x\textbackslash{}le\ 0\textbackslash{}\textbackslash{}\ x\ \&\ \textbackslash{}text\{si\ \}x\textbackslash{}ge\ 0\ \textbackslash{}end\{array\}\textbackslash{}right.\$\$}

\[|x|=\left\{\begin{array}{rr}
-x & \text{si }x\le 0\\
x & \text{si }x\ge 0
\end{array}\right.\]

La función \texttt{text\{\}} nos permite introducir texto en fórmulas
matemáticas.

\end{frame}

\hypertarget{paruxe1metros-de-los-chuncks-de-r}{%
\section{Parámetros de los chuncks de
R}\label{paruxe1metros-de-los-chuncks-de-r}}

\begin{frame}[fragile]{Chunks de R}
\protect\hypertarget{chunks-de-r}{}

Chunk. Bloque de código.

Los bloques de código de R dentro de un documento R Markdown se indican
de la manera siguiente

\begin{verbatim}

` x = 1+1`

` x`

```</div>

que resulta en


```r
x = 1+1
x
\end{verbatim}

\end{frame}

\begin{frame}{Chunks de R}
\protect\hypertarget{chunks-de-r-1}{}

Hay diversas opciones de crear un bloque de código de R:

\begin{itemize}
\tightlist
\item
  Ir al menú desplegable de ``Chunks'' y seleccionar el de R
\item
  Introducir manualmente
\item
  Alt + Command + I (para Mac) o Ctrl + Alt + I (para Windows)
\end{itemize}

\end{frame}

\begin{frame}[fragile]{Chunks de R}
\protect\hypertarget{chunks-de-r-2}{}

A los chunks se les puede poner etiqueta, para así localizarlos de
manera más fácil. Por ejemplo

```\{r PrimerChunk\}

\texttt{x\ =\ 1+2+3}

\begin{verbatim}

\n


<div class = "r-code">
```{r SegundoChunk}

` y = 1*2*3`

```</div>




## Parámetros de los chunks

La parte entre llaves también puede contener diversos parámetros, separados por comas entre ellos y separados de la etiqueta (o de r, si hemos decidido no poner ninguna).

Estos parámetros determinan el comportamiento del bloque al compilar el documento pulsando el botón `Knit` situado en la barra superior del área de trabajo.

## Parámetros de los chunks

Código |  Significado                                  
--------------------|--------------------
`echo` | Si lo igualamos a `TRUE`, que es el valor por defecto, estaremos diciendo que queremos que se muestre el código fuente del chunk. En cambio, igualado a `FALSE`, no se mostrará
`eval` | Si lo igualamos a `TRUE`, que es el valor por defecto, estaremos diciendo que queremos que se evalúe el código. En cambio, igualado a `FALSE`, no se evaluará
`message` | Nos permite indicar si queremos que se muestren los mensajes que R produce al ejecutar código. Igualado a `TRUE` se muestran, igualado a `FALSE` no
`warning` | Nos permite indicar si queremos que se muestren los mensajes de advertencia que producen algunas funciones al ejecutarse. Igualado a `TRUE` se muestran, igualado a `FALSE` no

## Parámetros de los chunks

<div class = "r-code">
```{r, echo=FALSE}

` sec = 10:20`

`sec`

`cumsum(sec)`

```</div>

\n

No aparece

## Parámetros de los chunks

<div class = "r-code">
```{r, echo=TRUE, message = TRUE}

`library(car)`

`head(cars,3)`

```</div>

\n


```r
library(car)
\end{verbatim}

\begin{verbatim}
## Loading required package: carData
\end{verbatim}

\begin{Shaded}
\begin{Highlighting}[]
\KeywordTok{head}\NormalTok{(cars,}\DecValTok{3}\NormalTok{)}
\end{Highlighting}
\end{Shaded}

\begin{verbatim}
##   speed dist
## 1     4    2
## 2     4   10
## 3     7    4
\end{verbatim}

\end{frame}

\begin{frame}[fragile]{Parámetros de los chunks}
\protect\hypertarget{paruxe1metros-de-los-chunks}{}

```\{r, echo=TRUE, message = FALSE, comment = NA\}

\texttt{library(car)}

\texttt{head(cars,3)}

\begin{verbatim}

\n


```r
library(car)
head(cars,3)
\end{verbatim}

\begin{verbatim}
  speed dist
1     4    2
2     4   10
3     7    4
\end{verbatim}

Fijaos que \texttt{comment=NA} evita que aparezcan los \texttt{\#\#}

\end{frame}

\begin{frame}[fragile]{Parámetros de los chunks}
\protect\hypertarget{paruxe1metros-de-los-chunks-1}{}

\begin{longtable}[]{@{}lll@{}}
\toprule
\begin{minipage}[b]{0.30\columnwidth}\raggedright
Significado\strut
\end{minipage} & \begin{minipage}[b]{0.30\columnwidth}\raggedright
Código\strut
\end{minipage} & \begin{minipage}[b]{0.30\columnwidth}\raggedright
Resultado\strut
\end{minipage}\tabularnewline
\midrule
\endhead
\begin{minipage}[t]{0.30\columnwidth}\raggedright
\texttt{results}\strut
\end{minipage} & \begin{minipage}[t]{0.30\columnwidth}\raggedright
\texttt{markup}\strut
\end{minipage} & \begin{minipage}[t]{0.30\columnwidth}\raggedright
Valor por defecto. Nos muestra los resultados en el documento final
línea a línea, encabezados por \texttt{\#\#}\strut
\end{minipage}\tabularnewline
\begin{minipage}[t]{0.30\columnwidth}\raggedright
\texttt{results}\strut
\end{minipage} & \begin{minipage}[t]{0.30\columnwidth}\raggedright
\texttt{hide}\strut
\end{minipage} & \begin{minipage}[t]{0.30\columnwidth}\raggedright
No se nos muestra el resultado en el documento final\strut
\end{minipage}\tabularnewline
\begin{minipage}[t]{0.30\columnwidth}\raggedright
\texttt{results}\strut
\end{minipage} & \begin{minipage}[t]{0.30\columnwidth}\raggedright
\texttt{asis}\strut
\end{minipage} & \begin{minipage}[t]{0.30\columnwidth}\raggedright
Nos devuelve los resultados línea a línea de manera literal en el
documento final y el programa con el que se abre el documento final los
interpreta como texto y formatea adecuadamente\strut
\end{minipage}\tabularnewline
\begin{minipage}[t]{0.30\columnwidth}\raggedright
\texttt{results}\strut
\end{minipage} & \begin{minipage}[t]{0.30\columnwidth}\raggedright
\texttt{hold}\strut
\end{minipage} & \begin{minipage}[t]{0.30\columnwidth}\raggedright
Miestra todos los resultados al final del bloque de código\strut
\end{minipage}\tabularnewline
\bottomrule
\end{longtable}

\end{frame}

\begin{frame}[fragile]{Parámetros de los chunks}
\protect\hypertarget{paruxe1metros-de-los-chunks-2}{}

```\{r, echo=TRUE, results = ``markup''\}

\texttt{sec\ =\ 10:20}

\texttt{sec}

\texttt{cumsum(sec)}

\begin{verbatim}

\n


```r
sec = 10:20
sec
\end{verbatim}

\begin{verbatim}
##  [1] 10 11 12 13 14 15 16 17 18 19 20
\end{verbatim}

\begin{Shaded}
\begin{Highlighting}[]
\KeywordTok{cumsum}\NormalTok{(sec)}
\end{Highlighting}
\end{Shaded}

\begin{verbatim}
##  [1]  10  21  33  46  60  75  91 108 126 145 165
\end{verbatim}

\end{frame}

\begin{frame}[fragile]{Parámetros de los chunks}
\protect\hypertarget{paruxe1metros-de-los-chunks-3}{}

```\{r, echo=TRUE, results = ``hide''\}

\texttt{sec\ =\ 10:20}

\texttt{sec}

\texttt{cumsum(sec)}

\begin{verbatim}

\n


```r
sec = 10:20
sec
cumsum(sec)
\end{verbatim}

\end{frame}

\begin{frame}[fragile]{Parámetros de los chunks}
\protect\hypertarget{paruxe1metros-de-los-chunks-4}{}

```\{r, echo=TRUE, results = ``asis''\}

\texttt{sec\ =\ 10:20}

\texttt{sec}

\texttt{cumsum(sec)}

\begin{verbatim}

\n


```r
sec = 10:20
sec
\end{verbatim}

{[}1{]} 10 11 12 13 14 15 16 17 18 19 20

\begin{Shaded}
\begin{Highlighting}[]
\KeywordTok{cumsum}\NormalTok{(sec)}
\end{Highlighting}
\end{Shaded}

{[}1{]} 10 21 33 46 60 75 91 108 126 145 165

\end{frame}

\begin{frame}[fragile]{Parámetros de los chunks}
\protect\hypertarget{paruxe1metros-de-los-chunks-5}{}

```\{r, echo=TRUE, results = ``hold''\}

\texttt{sec\ =\ 10:20}

\texttt{sec}

\texttt{cumsum(sec)}

\begin{verbatim}

\n


```r
sec = 10:20
sec
cumsum(sec)
\end{verbatim}

\begin{verbatim}
##  [1] 10 11 12 13 14 15 16 17 18 19 20
##  [1]  10  21  33  46  60  75  91 108 126 145 165
\end{verbatim}

\end{frame}

\hypertarget{los-chunks-en-modo-luxednea}{%
\section{Los chunks en modo línea}\label{los-chunks-en-modo-luxednea}}

\begin{frame}[fragile]{Los chunks en modo línea}
\protect\hypertarget{los-chunks-en-modo-luxednea-1}{}

Con lo explicado hasta ahora, solamente hemos generado resultados en la
línea aparte

Para introducir una parte de código dentro de un párrafo y que se
ejecute al comilarse el documento mostrando así el resultado final, hay
que hacerlo utilizando \texttt{\textasciigrave{}r\ ...\textasciigrave{}}

\textbf{Ejemplo}

La raíz cuadrada de 64 es
\texttt{\textasciigrave{}r\ sqrt(64)\textasciigrave{}} o, lo que viene
siendo lo mismo,
\(\sqrt{64}=\)\texttt{\textasciigrave{}r\ sqrt(64)\textasciigrave{}}

La raíz quinta de 32 es 2 o, lo que viene siendo lo mismo,
\(\sqrt[5]{64}=\) 2.

\end{frame}

\end{document}
